\documentclass[10pt]{article}
\usepackage[english]{babel}									
\usepackage[utf8]{inputenc}									
\usepackage[T1]{fontenc}										
\usepackage{amsmath,amsfonts,amssymb,amsthm,cancel,siunitx,
calculator,calc,mathtools,empheq,latexsym}
\usepackage{subfig,epsfig,tikz,float}		           
\usepackage{booktabs,multicol,multirow,tabularx,array}        
\usepackage{natbib}
\setlength{\parindent}{0pt}
\setlength{\parskip}{5pt}
\textwidth 13.5cm
\textheight 19.5cm
\columnsep .5cm
\title{\renewcommand{\baselinestretch}{1.17}\normalsize\bf%
\uppercase{<Put Title Here>}
}
\author{
<Put names here>
}


\begin{document}

\date{}

\maketitle

\vspace{-0.5cm}

\begin{center}
{\footnotesize 
<put class name, date, emails, and university affiliation here>
}
\end{center}
\bigskip
\noindent
{\small{\bf ABSTRACT.}
Abstract should concisely
summarize the key findings of the paper. It should consist 
of a single paragraph containing no more than 150 words. 
The Abstract does not have a section number.
}

\medskip
\noindent
{\small{\bf Keywords}{:} 
After the abstract three keywords must be provided.
}

\baselineskip=\normalbaselineskip
% -------------------------------------------------------------------

\section{Introduction and Background}\label{sec:1}

\section{Literature Review}\label{sec:2}
Logically, we assume that there is a non-linear relationship between the price of electricity and the weather. One assumption that we make is that in the summers, people living in warmer climates are more likely to use air conditioning. In the winters in areas that are cold, there are more likely people that are going to use a heater. In an article by Engle, Granger, Rice, and Weiss titled “Semiparametric Estimates of the Relation between Weather and Electricity Sales”, Engle et al. describe the various factors that affect the relationship between price and weather. 

\section{Data-set Description and Analysis}\label{sec:3}
In this section we describe the results...

\subsection{Case 1}
An example of subsection...

\section{Methodology}\label{sec:4}
We worked hard, and achieved very little...

\newpage

\section{Experimental Results}\label{sec:5}

Write all the results here!

\section{Conclusion and Discussion}\label{sec:6}


% FIM DO CONTEÚDO DO DOCUMENTO
% -----------------------------
\bibliographystyle{apalike} % Estilo de Bibliografia
\bibliography{referencias.bib} % Lista de Referências Bibliográficas

\end{document}
